\documentclass[a4paper,12pt]{report}
\usepackage[utf8]{inputenc}
\usepackage[right=30mm,left=30mm]{geometry}
\usepackage{microtype}
\usepackage[T1]{fontenc}
\usepackage{natbib}
\usepackage[francais]{babel}
\usepackage[Bjornstrup]{fncychap}
\usepackage{amssymb}
\usepackage{amsmath}
\usepackage{amsfonts}

\usepackage{xspace}

\usepackage{fourier}
%\usepackage{lmodern}

\usepackage{textcomp}

\usepackage{graphicx}
\usepackage{hyperref}
%\usepackage{pdflscape} %% mettre une page au format portrait (landscape)

\begin{document}
	\textbf{Slide 1}\\
Bonjour! Je vais faire la présentation de mon proposé de thèse qui porte sur l'effet des variations climatiques sur les propriétés du bois . La présentation va portée sur le contexte du sujet et les 3 chapitre de celui-ci.\\
	
	\textbf{Slide 2}\\	
L'espèce qui fait l'objet de notre étude est l'épinette blanche. Nous nous sommes intéressé à l'épinette blanche car c'est une espèce à large répartition en Amérique du nord avec une forte capacité d'adaptation. Sur l'image nous avons sa répartition en Amérique du nord. Nous allons nous intéressés aux propriétés de son bois dans le but d'améliorer la qualité du bois. Plusieurs études ont montré que les propriétés du bois sont regi par les conditions environnementales et par la génétique.\\  
	
	\textbf{Slide 3}\\
Ici nous avons une étude qui fait le lien entre la variabilité des familles et la densité du bois. Sur cette figure on vois que la densité du bois et la largeur des cernes peut varié en fonction de la famille.\\
	
	\textbf{Slide 4}\\
Comme dis précédemment le climat influence aussi les propriétés du bois. Les précipitation et les précipitations affectent aussi les propriétés du bois. Le bois est formé pendant l'activité cambiale; Et sur cette figure, on voie que l'augmentation ou la diminution de température influence l'activité cambiale. Plusieurs méthodes sont utilisé pour évaluer l'impact du climat sur les propriétés du bois. Parmi celles-ci on a l'isotopie\\

\textbf{Slide 5}\\
L'isotopie est utiliser pour évaluer l'impact du climat sur le bois. Mais pourquoi l'isotopie? L'isotopie est un outils précis pour l'évaluation des effets du climat sur les propriétés du bois.\\

\textbf{Slide 6}\\
Sur ce graphique qui extrait d'une étude de Drew, on observe une variation du 13c avec les périodes de sécheresse et cette variation est similaire à celle de la densité et de l'AMF. Ce qui explique le choix de la signature isotopique pour comprendre l'impact du climat sur les propriétés du bois.\\

\textbf{Slide 7}\\
Les objectifs de notre étude sont donc: 1, 2 et 3\\


\textbf{Slide 7}\\

























\end{document}